\documentclass[12pt]{article}

\usepackage[utf8]{inputenc}
\usepackage[T1]{fontenc}
\usepackage[font=small,format=plain,labelfont=bf,up,textfont=it,up]{caption}
\usepackage{xcolor}
\usepackage{hyperref}
\usepackage[framemethod=tikz]{mdframed}
\usepackage{pygmentex}


\definecolor{shadecolor}{rgb}{0.9,0.9,0.9}
\definecolor{lightgreen}{rgb}{0.8,0.95,0.8}

\def\a{Bla bla bla bla bla bla bla bla bla bla bla.}
\def\b{\a\a\a\a\a}

\begin{document}

\title{Testing Pygmen\TeX}
\author{José Romildo}
\maketitle

\section{The Pygmen\TeX{} package}

This document demonstrates how to use the pygment\TeX{} package to typeset
code listings with \LaTeX{} and
Pygments\footnote{\url{http://pygments.org/}}.

Pygments is a generic syntax highlighter for general use in all kinds of
software such as forum systems, wikis or other applications that need to
prettify source code.

Pygmen\TeX{} provides an environment and two commands for typesetting code
listings in a \LaTeX{} document:
\begin{itemize}
  \item the \texttt{pygmented} environment typesets its contents,
  \item the \texttt{includepygmented} command typesets the contents of a
  file, including the result in the \LaTeX{} document, and
  \item the \texttt{pyginline} command typesets its contents, keeping
  the result in the same line.
\end{itemize}
They accept many options that allow the user to configure the listing in
many ways.

Read the remaining of this document to have an ided of what the package
is capable of.

\section{How to use the package}

In order to use the package, start by putting
\begin{verbatim}
\usepackage{pygmentex}
\end{verbatim}
in the preamble of the document.

Use the environment or commands mentioned previously to include source
listings on your document.

When compiling the document (with \texttt{pdflatex}, for instance), all
the source listings in the document wil be collected and saved in a
temporary file with the extension \texttt{.snippets} in its name. Then
the auxiliary program \texttt{pygmentex} (a Python application
distributed with the pygmen\TeX{} package) should be run taking this
file as input. It will produce another temporary file with the extension
\texttt{.pygmented} containing \LaTeX{} code for the code listings
previously collected. The next time the document is compiled, they are
included to produce the final typeset document.

\section{First example}

See listing \ref{lst:a} for an example.

\begin{pygmented}[
  lang=scheme,
  linenos,
  backgroundcolor=shadecolor,
  label=lst:a,
  caption=A \textsl{Scheme} program.,
  ]
(define fact
    (lambda (n)
        (if (= n 0)
            1
            (* n (fact (- n 1))))))
\end{pygmented}


\section{Inline code snippets}

An inline source code snippet:
\pyginline[lang=c]|const double alfa = 3.14159;|
This is a C declaration with initialization.

\pyginline[lang=prolog,backgroundcolor=yellow]=avo(A,B) :- pai(A,X), pai(X,B).=
is a Prolog clause. Its head is
\pyginline[lang=prolog,style=emacs,backgroundcolor=yellow,linecolor=red]=avo(A,B)=
and its body is
\pyginline[lang=prolog,style=vim,backgroundcolor=black,hidealllines]=pai(A,X), pai(X,B)=.

\setpygmented{backgroundcolor=shadecolor}

This one
\pyginline[lang=ocaml,font=\ttfamily\scriptsize,topline=false]:let x = [1;2;3] in length x:
is an OCaml expression with local bindings. With OCaml one can do
imperative, functional and object oriented programming.

Now some Java code:
\pyginline[lang=java,style=colorful,font=\ttfamily\itshape,linewidth=1pt]|public int f(double x)|.
This is a method header.

\section{Displayed code snippets}

\setpygmented{lang=scheme,backgroundcolor=shadecolor,linecolor=blue,style=emacs}

In listing \ref{lst:fact} you can see a function definition in the
Scheme language. This function computes the factorial of a natural
number.  \newline\rule{\linewidth}{2pt}
\begin{pygmented}[
  style=emacs,
  rightline=false,
  linenos,
  label=lst:fact,
  caption=A \textbf{Scheme} function.
  ]
(define fact
    (lambda (n)
        (if (= n 0)
            1
            (* n (fact (- n 1))))))
\end{pygmented}

Here you have some more code to further testing the package. Listing
\ref{lst:haskell} is a Haskell program. When run this program interacts
with the user asking the user name, reading a line input by the user,
and showing a greeting message to the user.

\inputpygmented[
lang=haskell,
linenos,
linenostart=79831,
innerlinecolor=yellow, innerlinewidth=6pt,
middlelinecolor=blue, middlelinewidth=10pt,
outerlinecolor=green, outerlinewidth=12pt,
roundcorner=4,
backgroundcolor=shadecolor,
caption=A haskell interactive program,
label=lst:haskell,
]{demo.hs}

This is a rule:

\noindent\rule{\linewidth}{2pt}

Now a Pascal procedure:

\inputpygmented[
  lang=delphi,
  linewidth=1.5pt,
  font=\ttfamily\sffamily\large,
  backgroundcolor=yellow
  ]{demo.delphi}
and a Pascal program
\inputpygmented[lang=pascal,linenos,linenostart=5801]{demo.pas}

A Python code snippet:

\inputpygmented[lang=python,style=emacs,linenos,linenostep=3,linewidth=1pt,backgroundcolor=lightgreen]{demo.py}

\section{Escaping to \LaTeX{} in code snippets}

\begin{pygmented}[lang=c,style=manni,escapeinside=||,
% caption=Conditional command.
]
if (|\textit{condition}|)
  |\textit{command$_1$}|
else
  |\textit{command$_2$}|
\end{pygmented}


\section{Using code snippets in environments}

The following is a \textbf{description} environment.

\begin{description}
  \item[An item] \a
  \begin{pygmented}[lang=scala,backgroundcolor=yellow,
    % title=Item A
    ]
def qsort(xs: List[Int]): List[Int] =
  xs match {
    case Nil =>
      Nil
    case pivot :: tail =>
      qsort(tail filter { _ < pivot }) :::
        pivot :: qsort(tail filter { _ >= pivot })
  }
  \end{pygmented}
  \b
  
  \item[Another item] \a
  \begin{pygmented}[lang=lua,backgroundcolor=yellow]
--- Spawn a program.
-- @param cmd The command.
-- @param sn Enable startup-notification.
-- @param screen The screen where to spawn window.
-- @return The awesome.spawn return value.
function spawn(cmd, sn, screen)
    if cmd and cmd ~= "" then
        if sn == nil then sn = true end
        return capi.awesome.spawn(cmd, sn, screen or capi.mouse.screen)
    end
end
  \end{pygmented}
  \b
\end{description}

\section{A long program}

Here you can read the source code for the \texttt{pygmentex.py} program
that I have written.

\inputpygmented[
lang=python,
style=autumn,
backgroundcolor=green!10,
font=\ttfamily\scriptsize,
]{pygmentex.py}

\section{Conclusion}

That is all.


\end{document}
